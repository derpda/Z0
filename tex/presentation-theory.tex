% \begin{frame}
% 	\frametitle{Wirkungsquerschnitt}
% 	\begin{figure}
% 	\begin{center}
% 	  \includegraphics[width=0.3\textwidth]{../img/atlas_higgs_event.png}
% 	  \caption{Produkte einer Proton-Proton-Kollision beim ATLAS-Experiment am CERN (Quelle: https://cds.cern.ch/record/1459496)}
% 	\end{center}
% 	\end{figure}
% 	\begin{itemize}
% 		\item Bei Streuprozessen ist der Endzustand nicht eindeutig bestimmt
% 		\item Der Wahrscheinlichkeit eines bestimmten Endzustandes wird durch den Wirkungsquerschnitt $\sigma$ beschrieben.
% 		\item Differentieller Wirkungsquerschnitt $\frac{\difd \sigma}{\difd O}$ in Bezug auf \\
% 			  Observable $O$ (z.B. Raumwinkel $\Omega$, Transversalimpuls $p_\text{T}, \ldots$)
% 	\end{itemize}
% \end{frame}

\section{Theoretische Grundlagen}

\subsection{Das Standardmodell}

\subsection{Elektroschwache Wechselwirkung}

\subsection{Wirkungsquerschnitt und Zerfallsbreite}

\subsection{$e^+e^-$ Kollisionen}

\subsection{Strahlungskorrektur}

\subsection{Vorwärts Rückwärts Asymmetrie}