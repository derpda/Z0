\section{Data analysis}
As no data was surveyed by the authors, there will be no chapter on the experimental procedure. Instead, the focus will be on the analysis of the data, for most of which the framework ROOT was used. Data was provided from two sources:
\begin{enumerate}
	\item Monte Carlo simulations, separated by decay reaction.
	\item Actual data from the OPAL detector at the LEP collider.
\end{enumerate}


\subsection{Selection of cuts}
In a first step, the data from the Monte Carlo simulations is plotted over adequate range in the following parameters:
\begin{itemize}
	\item{\makebox[3cm][l]{\textbf{NCHARGED:}}The number of tracks visible in the drift chamber}
	\item{\makebox[3cm][l]{\textbf{PCHARGED:}}Energy of particles that left a track in the drift chamber}
	\item{\makebox[3cm][l]{\textbf{E\_ECAL:}}Energy deposited in the electromagnetic calorimeter}
	\item{\makebox[3cm][l]{\textbf{E\_HCAL:}}Energy deposited in the hadronic calorimeter}
	\item{\makebox[3cm][l]{\textbf{COS\_THET:}}Angle $\theta$ between created the positive lepton and the \\\makebox[3cm][l]{}incident positron beam}
	\item{\makebox[3cm][l]{\textbf{COS\_THRU:}}Angle between the thrust axis for hadronic events and the \\\makebox[3cm][l]{}incident positron beam}
\end{itemize}
These plots include the simulation events from decays to electrons, myons, tauons as well as quarks in similar amounts. These ratios are not a representation of the actual ratios as they would be expected in the real data, as the decay width of quarks is much greater than that of the three types of leptons. \\
To further improve comparability of the different data sets, all histograms were normed to an integral of 1. The resulting plots are shown in graphics !!ADD GRAPHS REF!!. Using these graphs, as well as the knowledge of the type of event, cuts are established which will later be used to separate the actual data, where there is no initial knowledge of the type of event.\\
Thus, for every type of educts, a cut is to be made with maximum possible efficiency in detecting the respective kind of event, as well as maximum purity, meaning that other events are falsely allotted as rarely as possible.

\newpage
\begin{figure}[H]
\centering
\includegraphics[width=1\linewidth]{../results/MC_results/nocut/Ncharged}
\caption[Ncharged in simulation data]{The number of charged tracks in the simulation data for the various decay events (color-coded). Almost all muon events leave two tracks. The peak is cut off to increase visibility of the other data.}
\label{fig:Ncharged}
\end{figure}

\subsubsection{Cuts in Ncharged}
Above figure suggests a good way to separate the data sets. In particular, all three lepton generations hardly ever leave more than 6 cuts, which then suggests the following cuts:

\begin{itemize}
	\item{\makebox[2.5cm][l]{\textbf{Lepton cuts:}} Ncharged $<7$}
	\item{\makebox[2.5cm][l]{\textbf{Quark cut:}} Ncharged $\ge8$}
\end{itemize}



\newpage
\begin{figure}[H]
\centering
\includegraphics[width=1\linewidth]{../results/MC_results/nocut/Pcharged}
\caption[Pcharged in simulation data]{This figure shows the sum of track energies in the simulation data. }
\label{fig:Pcharged}
\end{figure}

\subsubsection{Cuts in Pcharged}
The separation of the data sets is less obvious in the Pcharged channel. Most datasets overlap, preventing the very clean cuts as they were in the Ncharged channel. However, since we have conveniently already cut against quarks in the Ncharged channel, we can ignore said dataset and separate tauons and muons as follows:
\begin{itemize}
	\item{\makebox[2.5cm][l]{\textbf{Muon cut:}} Pcharged $\le60$ GeV}
	\item{\makebox[2.5cm][l]{\textbf{Tauon cut}} Pcharged $>71$ GeV}
\end{itemize}

\newpage
\begin{figure}[H]
\centering
\includegraphics[width=1\linewidth]{../results/MC_results/nocut/E_Ecal}
\caption[E\_Ecal in simulations]{The energies deposited in the electronic calorimeter show clear differences for the four decay types and allow for effective cuts.}
\label{fig:E_Ecal}
\end{figure}

\subsubsection{Cuts in E\_Ecal}
As has been the case for the cuts in Pcharged, we can ignore the quark data. Instead, cuts for all leptons are applied in order to separate the three kinds:

\begin{itemize}
	\item{\makebox[2.5cm][l]{\textbf{Electron cut}} E\_Ecal $\ge70$ GeV}
	\item{\makebox[2.5cm][l]{\textbf{Muon cut:}} E\_Ecal $<50$ GeV}
	\item{\makebox[2.5cm][l]{\textbf{Tauon cut:}} E\_Ecal $<60$ GeV}
\end{itemize}

\newpage
\begin{figure}[H]
\centering
\includegraphics[width=1\linewidth]{../results/MC_results/nocut/E_Hcal}
\caption[E\_Hcal in simulation data]{The plot shows the energies deposited in the hadronic calorimeter. The distribution of the decay types overlap significantly.}
\label{fig:E_Hcal}
\end{figure}

\subsubsection{Cuts in E\_Hcal}
In the data from the hadronic calorimeter, the overlap between all four kinds of decay events does not allow for any kind of meaningful cuts.

\newpage
\begin{figure}[H]
\centering
\includegraphics[width=1\linewidth]{../results/MC_results/nocut/cos_theta}
\caption[Cos\_theta in simulation data]{This figure shows the cosine of the angle between the beam and the direction of the created anti-lepton. Note the asymmetric peaks in the electron data, which will be discussed in the chapter on the separation of s- and t-channel.}
\label{fig:cos_theta}
\end{figure}

\subsubsection{Cuts in Cos\_theta}
As can be seen in figure \ref{fig:cos_theta}, the quark events are not assigned values of Cos\_theta within its natural boundaries from -1 to 1. This is due to the fact that they cause scattering jets, meaning that a well defined direction cannot be assigned. Leptonic events can also cause the creation of more than two particles. Furthermore, as there is no detection capacity in beam direction as well as for a very shallow angles, some regular events are also not assigned an angle. As a consequence, such events are listed with a Cos\_theta of $999.0$.\\
Since the electron data is later divided into s- and t-channel events, for which the Cos\_theta distribution is needed, it has to be ensured that the data is indeed valid. As there is no detection capacity for shallow angles to the beam direction, such events are cut off:

\begin{itemize}
	\item{\makebox[2.5cm][l]{\textbf{Electron cut}} $-0.9\le$ cos\_theta $\le0.9$}
\end{itemize}

\newpage
\begin{figure}[H]
\centering
\includegraphics[width=1\linewidth]{../results/MC_results/nocut/cos_thru}
\caption[Cos\_thru in simulation data]{This figure shows the cosine of the angle between the beam and the thrust direction.}
\label{fig:cos_thru}
\end{figure}

\subsubsection{Cuts in Cos\_thrust}
Figure \ref{fig:cos_thru} suggests a cut to separate tauons from the rest of the particles, especially electrons, which often have have $cos_thrust>0.9$ or $cos_thrust<-0.9$

\begin{itemize}
	\item{\makebox[2.5cm][l]{\textbf{Tauon cut}} $-0.9\le$ cos\_thrust $\le0.9$}
\end{itemize}

The final cuts thus are
\begin{table}[H]\centering
	\begin{tabular}{@{}llllllll@{}}
		\toprule
		&			&Ncharged	&Pcharged [GeV]	&E\_Ecal [GeV] &Cos\_theta				&Cos\_thrust\\ 
		\midrule
		&$e^+e^-$	&			&				&$\ge70$		&$\ge-0.9$ \& $\le0.9$	&\\
		&$\mu^+\mu^-$		&			&$\le60$		&$<50$			&						&\\
		&$\tau^+\tau^-$		&$<7$		&$>71$			&$<60$			&						&\\
		&$q\overline{q}$		&$\ge8$		&				&				&						&$\ge-0.9$ \& $\le0.9$	\\
		\bottomrule
	\end{tabular}
	\caption[Table of cuts]{Cuts applied to separate the datasets. The Pcharged $\ne0$ cut is not shown here but was applied to all datasets.}
\end{table}

\subsection{Two-photon events TODOTODOTODOTODOTODOTODO}
One possible source of background are two-photon events. Figure \ref{fig:twophotonfeynman} shows possible Feynman-graphs of such events. Two photons are created, without an intermediate Z$^0$. As such, these measurements need to be excluded using cuts.\\
Since these events send out either two pairs of fermions or no fermions at all into the detector, proper values of cos\_theta and cos\_thrust cannot be assigned. These are then automatically set to $999.00$. Thus, the applied cuts in those two variables already effectively filter against many of these events. One exception however would be, if one of the created photons in an event such as the left one in figure \ref{fig:twophotonfeynman} causes pair production. In such 
\begin{figure}[H]
\centering
\includegraphics[width=1.0\linewidth]{graphics/twophotonfeynman_both.pdf}
\caption[Two-photon Feynman-graph]{Two examplary Feynman-graphs of two-photon events. A Z$^0$ is never created, which means that these event should not contribute to our measurement.}
\label{fig:twophotonfeynman}
\end{figure}

\newpage
\subsection{Purity and the efficiency matrix}
These cuts are now applied to the simulated data, where the kind of event is known. This way, one can determine the efficiency and purity of the cuts and judge how well they work. 
The efficiency matrix is relatively easily calculated as
\begin{equation}
E_{ij}=\frac{n^{cut}_{ij}}{N_i}
\end{equation}
where the indexes $i$ and $j$ represent the kind of event. $n_{ij}^{cut}$ is thus the amount of events of type $i$ left over after applying the cut for type $j$ and $N_i$ is the overall number of events $i$ in the simulation data.
\begin{table}[H]\centering
	\begin{tabular}{@{}llllll@{}}
		\toprule
		&Events &$e^+e^-$&$\mu^+\mu^-$&$\tau^+\tau^-$&$q^+q^-$\\
		\midrule
		&Cut&&&&\\
		&$e^+e^-$&0.39090&0.00002&0.00442&0.00002\\
		&$\mu^+\mu^-$&0.00018&0.90171&0.00611&0.00001\\
		&$\tau^+\tau^-$&0.00039&0.00945&0.83877&0.00096\\
		&$q\overline{q}$&0.00007&0.00001&0.00687&0.98970\\
	\end{tabular}
	\caption[Efficiency matrix]{Efficiency matrix of the applied cuts. The diagonal elements are the self efficiency.}
	\label{tb:efficiency}
\end{table}

As these values describe event counts, they should be distributed binomially. According to Paterno \cite{binpaper}, the binomial error of such values is commonly calculated as
\begin{equation}
s_{E_{ij}}=\sqrt{\frac{E_{ij}\cdot(1-E_{ij})}{N_i}}
\end{equation}
where $N_i$ is the number of simulated events of the appropriate kind. This yields the following matrix

\begin{table}[H]\centering
	\begin{tabular}{@{}llllll@{}}
		\toprule
		&Events &$e^+e^-$&$\mu^+\mu^-$&$\tau^+\tau^-$&$q^+q^-$\\
		\midrule
		&Cut&&&&\\
		&$e^+e^-$&0.001593&0.000015&0.000236&0.000014\\
		&$\mu^+\mu^-$&0.000044&0.000969&0.000277&0.000010\\
		&$\tau^+\tau^-$&0.000065&0.000315&0.001307&0.000099\\
		&$q\overline{q}$&0.000028&0.000011&0.000293&0.000322\\
		\bottomrule
	\end{tabular}
	\caption[Efficiency error matrix]{Errors of the efficiency matrix elements calculated under the assumption of binomial distribution.}
	\label{tb:efficiencyerr}
\end{table}

To calculate the purity, it has to be taken into account that the different events do not occur with the same probability in nature but are roughly equally represented in the simulated data. This is expressed by the branching ratio
\begin{equation}
BR_i=\frac{\Gamma_i}{\sum_{i,j}\Gamma_{i}}
\end{equation}
where $\Gamma_i$ is the partial decay width corresponding to event $i$. The partial decay widths of leptons $\Gamma_l=\unit{83.8}{\mega\electronvolt}$ and of quarks $\Gamma_q=\unit{1732}{\mega\electronvolt}$ are given in \cite{staatsex} without error.
Using this, the purity can be calculated as
\begin{equation}
P_i=\frac{BR_i\cdot E_{ii}}{\sum_{j}BR_j\cdot E_{ij}}
\end{equation}
which yields the following results:

\begin{table}[H]\centering
	\begin{tabular}{@{}lll@{}}
		\toprule
		&Cut&Purity\\
		\midrule
		&$e^+e^-$&0.9877\\
		&$\mu^+\mu^-$&0.9928\\
		&$\tau^+\tau^-$&0.9657\\
		&$q\overline{q}$&0.9996\\
		\bottomrule
	\end{tabular}
	\caption[Purity of the cuts]{Purity of the cuts. All purities are above 95\%, indicating that the cuts work reasonably well.}
	\label{tb:purity}
\end{table}

\subsection{Calculation of the inverse efficiency matrix}
The efficiency matrix allows the calculation of the number of events after applying the cuts to a set of measurements where the kind of event is known:

\begin{equation}
\vec{C}=\boldsymbol{E}\cdot\vec{N}
\end{equation} 

where $\vec{N}$ is a vector whose components are the numbers of events of the four different kinds and $\vec{C}$ is a vector whose components are the number of events allocated to the four different cuts. 
In reality however, the kind of event is unknown and has to be determined by the cuts. Thus, the inverse of $\boldsymbol{E}$ can be used to calculate the number of actual events from the known number of events after the different cuts:
\begin{equation}
\vec{N}=\boldsymbol{E}^{-1}\cdot\vec{C}=:\boldsymbol{I}\cdot\vec{C}
\label{eq:numberevents}
\end{equation} 
The inverse matrix $\boldsymbol{I}$ is
\begin{table}[H]\centering
	\begin{tabular}{@{}llllll@{}}
		\toprule
		&Events &$e^+e^-$&$\mu^+\mu^-$&$\tau^+\tau^-$&$q^+q^-$\\
		\midrule
		&Cut&&&&\\
		&$e^+e^-$&2.55823&0.00008&-0.01348&-0.00004\\
		&$\mu^+\mu^-$&-0.00051&1.10909&-0.00808&-0.00000\\
		&$\tau^+\tau^-$&-0.00120&-0.01250&1.19233&-0.00116\\
		&$q\overline{q}$&-0.00018&0.00007&-0.00827&1.01041\\
		\bottomrule
	\end{tabular}
	\caption[Inverse efficiency matrix]{The inverse of the efficiency matrix displayed in table \ref{tb:efficiency}.}
	\label{tb:invefficiency}
\end{table}

To calculate the error of the inverse matrix, a method called \emph{toy experiments} has to be used. A normal-distribution random variable $G=\mathcal{N}(0,1)$ is multiplied by the error of an element of the efficiency matrix and then added to said element.
\begin{equation}
E^{k}_{ij}=E_{ij}+G^k\cdot s_{E_{ij}}
\end{equation}
This is done $N=100000$ times, resulting in 100000 randomly varied efficiency matrix, which are then inverted. The standard deviation of the thus calculated elements $I^k_{ij}$ of the inverse matrix is used as the error for the inverse matrix elements $I_{ij}$: 
\begin{equation}
s_{I_{ij}}=\sqrt{\frac{1}{N-1}\sum_{k=1}^{N}\left(I^k_{ij}-I_{ij}\right)^2}
\end{equation}
The resulting matrix is displayed in table \ref{tb:invefficiencyerror}
\begin{table}[H]\centering
	\begin{tabular}{@{}llllll@{}}
		\toprule
		&Events &$e^+e^-$&$\mu^+\mu^-$&$\tau^+\tau^-$&$q^+q^-$\\
		\midrule
		&Cut&&&&\\
		&$e^+e^-$&2.558226&0.000081&-0.013476&-0.000039\\
		&$\mu^+\mu^-$&-0.000506&1.109092&-0.008077&-0.000003\\
		&$\tau^+\tau^-$&-0.001197&-0.012497&1.192334&-0.001161\\
		&$q\overline{q}$&-0.000185&0.000075&-0.008272&1.010413\\
		\bottomrule
	\end{tabular}
	\caption[Inverse efficiency error matrix]{Errors of the inverse efficiency matrix (table \ref{tb:invefficiency}) calculated using the \emph{toy experiments} method.}
	\label{tb:invefficiencyerror}
\end{table}

\subsection{Separation of s- and t-channel}
\subsection{Cross sections and the mass of Z$^0$}
With the data cleared of underground effects, the cross sections can be calculated. The number of events after the respective cuts $\vec{C}$ are Poisson-distributed, thus
\begin{equation}
s_{C_i}=\sqrt{C_i}
\end{equation}
With equation \ref{eq:numberevents}, the error on the events $\vec{N}$ that presumably occurred can be calculated as
\begin{equation}
s_{N_i}=\sqrt{\sum_{i,j}\left((C_j\cdot s_{I_{ij}})^2+(I_{ij}\cdot s_{M_j})^2\right)}
\end{equation}
For electron-positron events, the s-t-channel separation has yet to be implemented:
\begin{equation}
N'_{e^+e^-}=c_{st}\cdot N_{e^+e^-}, \qquad s_{N'_{e^+e^-}}=\sqrt{(c_{st}\cdot s_{N_{e^+e^-}})^2+(N_{e^+e^-}\cdot s_{c_{st}})^2}
\end{equation}
In order to calculate the cross sections, the radiation corrections and the luminosity of the accelerator are needed. Both were provided and are listed in the following table

\begin{table}[H]\centering
	\begin{tabular}{@{}llllllll@{}}
		\toprule
		  &$\sqrt{s}$ [GeV]	&L [\nicefrac{1}{nb}]	&$s_L^{stat}$[\nicefrac{1}{nb}]	&$s_L^{sys}$[\nicefrac{1}{nb}]	&$s_L^{total}$[\nicefrac{1}{nb}]&$c_{beam,l^+l^-}$ [nb]&$c_{beam,q\overline{q}}$ [nb]\\
		  \midrule
		  &88.48	&675.9	&3.5	&4.5	&5.7	&0.09	&2.0\\  
		  &89.47	&543.6	&3.2	&3.6	&4.8	&0.20	&4.3\\  
		  &90.23	&419.8	&2.8	&2.8	&4.0	&0.36	&7.7\\  
		  &91.23	&3122.2	&7.8	&20.9	&22.3	&0.52	&10.8\\  
		  &91.97	&639.8	&3.6	&4.2	&5.6	&0.22	&4.7\\  
		  &92.97	&479.2	&3.1	&3.2	&4.5	&-0.01	&0.2\\  
		  &93.72	&766.8	&4.0	&5.1	&6.5	&-0.08	&-1.6\\
		\bottomrule
	\end{tabular}
	\caption[Table of luminosities]{The luminosities L and radiation corrections $c_{beam}$ for the different mean energies $\sqrt{s}$. The radiation corrections where found in \cite{anleitung}, where no error was given.}
	\label{tb:luminosity}
\end{table}

The cross section and their error can then be calculated as
\begin{equation}
\sigma_i=\frac{N_i}{L}+c_{beam,i}, \qquad s_{\sigma_i}=\sqrt{(\frac{s_{N_i}}{L})^2+(\frac{N_i\cdot s_L^{total}}{L^2})^2}
\end{equation}

Figures \ref{fig:WQee} - \ref{fig:WQqq} show the results of the calculations. The Breit-Wigner function from equation \ref{eq:principles:breitwigner} was fitted to the data.

\begin{figure}[H]
\centering
\includegraphics[width=1.0\linewidth]{../results/data_results/wqs/WQee}
\caption[Cross sections for electron cut]{Cross sections of the different $\sqrt{s}$ in the electron cut.}
\label{fig:WQee}
\end{figure}

\begin{figure}
\centering
\includegraphics[width=1.0\linewidth]{../results/data_results/wqs/WQmm}
\caption[Cross sections for muon cut]{Cross sections of the different $\sqrt{s}$ in the muon cut.}
\label{fig:WQmm}
\end{figure}

\begin{figure}
	\centering
	\includegraphics[width=1.0\linewidth]{../results/data_results/wqs/WQTT}
	\caption[Cross sections for tauon cut]{Cross sections of the different $\sqrt{s}$ in the tauon cut.}
	\label{fig:WQTT}
\end{figure}

\begin{figure}
\centering
\includegraphics[width=1.0\linewidth]{../results/data_results/wqs/WQqq}
\caption[Cross sections for quark cut]{Cross sections of the different $\sqrt{s}$ in the quark cut.}
\label{fig:WQqq}
\end{figure}

From fits yielded the following values for the mass of the Z$^0$-boson
\begin{table}[H]\centering
	\begin{tabular}{@{}llll@{}}
		\toprule
		&Event type&$M_{Z^0}$ [GeV]&$s_{M_{Z^0}}$ [GeV]\\
		\midrule
		&$e^+e^-$&91.20&0.03\\
		&$\mu^+\mu^-$&91.11&0.06\\
		&$\tau^+\tau^-$&91.18&0.04\\
		&$q\overline{q}$&91.19&0.009\\
		\midrule
		&Weighted mean&91.189&0.008\\
		&Literature value&91.1876&0.0021\\
		\bottomrule
	\end{tabular}
	\caption[Fitresults for $M_{Z^0}$]{The mass of Z$^0$ as it was calculated from the fits for the different cuts as well as their weighted mean.}
	\label{tb:Z0massfitresults}
\end{table}



\subsection{Analysis of the decay widths}
\subsection{Forward-backward asymmetry}


