\section{Figures}
\subsection{Cross section fit graphs}
\begin{figure}[H]
	\centering
	\includegraphics[width=1.0\linewidth]{../results/data_results/wqs/WQee}
	\caption[Cross sections for electron cut]{Cross sections of the different $\sqrt{s}$ in the electron cut.}
	\label{fig:WQee}
\end{figure}

\begin{figure}[H]
	\centering
	\includegraphics[width=1.0\linewidth]{../results/data_results/wqs/WQTT}
	\caption[Cross sections for tauon cut]{Cross sections of the different $\sqrt{s}$ in the tauon cut.}
	\label{fig:WQTT}
\end{figure}

\begin{figure}[H]
	\centering
	\includegraphics[width=1.0\linewidth]{../results/data_results/wqs/WQqq}
	\caption[Cross sections for quark cut]{Cross sections of the different $\sqrt{s}$ in the quark cut.}
	\label{fig:WQqq}
\end{figure}

To not assume lepton universality, the values can be calculated from the fit parameters as follows
\begin{equation*}
	\Gamma_i=\frac{\Gamma_f^2}{\Gamma_e}
\end{equation*}
\begin{table}[h]\centering
	\begin{tabular}{@{}lllllll@{}}
		\toprule
		&Event $i$&$\Gamma_{Z}$ [GeV]&$s_{\Gamma_{Z}}$ [GeV]&$\Gamma_i$ [MeV]&$s_{\Gamma_i}$ [MeV]&$\Gamma_i^{lit}$ [MeV]\\
		\midrule
		&$e^+e^-$&2.28&0.11&94&4&83.8\\
		&$\mu^+\mu^-$&2.52&0.06&73&4&83.8\\
		&$\tau^+\tau^-$&2.48&0.07&63&4&83.8\\
		&$q\overline{q}$&2.526&0.019&1570&70&1732\\
		\bottomrule
	\end{tabular}
	\caption[Decay widths not assumint universality]{The decay widths when not assuming universality. Clearly, the results are much worse.}
	\label{tb:appendix:decaywidthsnouniv}
\end{table}	